\section{Boolean Properties of Sets}

\begin{xca}
Let $\psi$ be any proposition. Write down the truth table for
$\bot\implies\psi$. You should find this is a tautology (i.e., it is
always true).
\end{xca}

\begin{definition}
Let $X$ be a set. We say $X$ is \define{empty} to mean
\begin{equation}\label{eq:bool:def:empty}
\mbox{for any object $x$, we have }x\notin X.
\end{equation}
\end{definition}

\begin{xca}\label{xca:bool:empty-is-adjective}
Is this defining an adjective or a predicate? Why?
\end{xca}

\begin{proposition}\label{prop:bool:existence-of-empty-set}
There exists an empty set.
\end{proposition}

We will use the Scheme~\ref{sch:bool:replacement} and the axiom of
choice (``Choose some set $A$ --- we know at least one exists, so choose
it.'') The predicate will be $\mathcal{P}[x]=\bot$ constantly false. So
what elements of $A$ satisfy this? None at all. This allows us to
construct the set $X$ consisting of elements of $A$ which satisfy an
impossible-to-satisfy predicate. Since there are no such elements of $A$,
it follows that $X$ is empty.

\begin{proof}
We define the unary predicate of sets $\mathcal{P}[x]=\bot$.
We define the constant set $\mathcal{A}$ to be selected by the axiom of
choice. 
Using Scheme~\ref{sch:bool:replacement}, consider the set $X$ such that
\begin{subequations}
\begin{equation}
\mbox{for all objects $x$, we have $x\in X$}\iff\mbox{$x\in\mathcal{A}$ and $\mathcal{P}[x]$}.
\end{equation}
Now we unfold the definition of $\mathcal{P}[x]$
\begin{equation}
\mbox{for all objects $x$, we have $x\in X$}\iff\mbox{$x\in\mathcal{A}$ and $\bot$}.
\end{equation}
We use the logical equivalence $(A\land\bot)\iff\bot$,
\begin{equation}
\mbox{for all objects $x$, we have $x\in X$}\iff\bot.
\end{equation}
This is equivalent to the conjunction of $(x\in X)\implies\bot$ (which is $\neg(x\in X)$ or
$x\notin X$) and $\bot\implies(x\in X)$ (which is vacuously true,
because ``contradiction implies anything'', so we can discard this clause).
Therefore this is equivalent to the statement
\begin{equation}
\mbox{for all objects $x$, we have }x\notin X.
\end{equation}
Then by using Eq~\eqref{eq:bool:def:empty}, we have
\begin{equation}\label{eq:bool:conclusion-of-existence-of-empty-set}
X\mbox{ is empty}.
\end{equation}
\end{subequations}
Take $X$, and we conclude from Eq~\eqref{eq:bool:conclusion-of-existence-of-empty-set} that there exists an empty set.
\end{proof}

We will prove there exists exactly one empty set. Towards that end, we
will prove some \emph{lemmas}. A ``lemma'' is a proposition which is
used as a stepping stone towards a bigger result. The roadmap is if we
want to prove there is a unique empty set, this means that for any two
empty sets $A_{1}$ and $A_{2}$, then $A_{1}=A_{2}$. We have already
established the existence of at least one empty set. (The bigger picture
is we will define a term $\emptyset$ for the empty set.)

\begin{lemma}\label{lemma:bool:empty-set-is-subset-of-any-set}
Let $A$ be an empty set, let $B$ be any set (possibly empty, possibly nonempty). Then $A\subset B$.
\end{lemma}

What do we need to prove? Well, we are trying to prove one set is a
subset of another, so we need to recall Eq~\eqref{eq:def:subset} for the
definitional theorem for subsets. The goal is to prove ``For all objects
$x$, if $x\in A$ then $x\in B$''.

\begin{proof}
Let $A$ be an empty set.
Let $B$ be any set. [We will prove ``For any object $x$, if $x\in A$,
then $x\in B$''.] Let $x$ be an arbitrary object.
Assume
\begin{equation}\label{eq:bool:pf-step:empty-set-is-contained-in-anything}
x\in A.
\end{equation}
Now we want to prove $x\in B$. [What do we do?
We have only one trick in our toolkit: return to the definition. Let us
return to the definition of ``empty'' found in Eq~\eqref{eq:bool:def:empty}.]
However, by Eq~\eqref{eq:bool:def:empty}, for any object $y$, we have
$y\notin A$. This contradicts Eq~\eqref{eq:bool:pf-step:empty-set-is-contained-in-anything},
and we can conclude $x\in B$ as desired.
\end{proof}

\begin{theorem}\label{thm:bool:uniqueness-of-empty-set}
For any empty sets $A$ and $B$, we have $A=B$.
\end{theorem}

We are trying to prove two sets are equal. Towards that end, we will use
the criteria established in Proposition~\ref{prop:subsets:equiv-set-equality}
to have our proof obligation be ``$A\subset B$ and $B\subset A$''.
We use the previous lemma twice.

\begin{proof}
Let $A$ and $B$ be empty sets. Now we will prove $A\subset B$ and
$B\subset A$, then invoke Proposition~\ref{prop:subsets:equiv-set-equality}
to conclude $A=B$.

We see $A\subset B$ by Lemma~\ref{lemma:bool:empty-set-is-subset-of-any-set}
and the fact that $A$ is empty.

We see $B\subset A$ by Lemma~\ref{lemma:bool:empty-set-is-subset-of-any-set}
and the fact that $B$ is empty.

Therefore by Proposition~\ref{prop:subsets:equiv-set-equality}
we conclude $A=B$.
\end{proof}

\begin{definition}
We define the term $\emptyset$ to be the set such that $\emptyset$ is
the empty set.
\end{definition}

We would normally need to prove the existence and uniqueness of
$\emptyset$, but existence was proven in
Proposition~\ref{prop:bool:existence-of-empty-set} and 
uniqueness was proven in Theorem~\ref{thm:bool:uniqueness-of-empty-set}.
Therefore, the term $\emptyset$ is well-defined.

\begin{xca}\label{xca:bool:subsets-of-empty-set-are-empty}
Prove or find a counter-example: for any set $X$, if $X\subset\emptyset$,
then $X=\emptyset$.
\end{xca}

\begin{xca}\label{xca:bool:modus-barbara}
Prove the \textit{Modus Barbara} syllogism: for any sets $X$, $Y$, and
$Z$, if $X\subset Y$ and $Y\subset Z$, then $X\subset Z$.
\end{xca}

