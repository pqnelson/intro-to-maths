\section{Unions of Sets}

\begin{definition}
Let $X$ and $Y$ be sets.
We define the term $X\cup Y$ (pronounced ``The union of $X$ and $Y$'')
to be the set satisfying
\begin{equation}\label{eq:def:unions:cup}
\mbox{For all objects $x$, }x\in X\cup Y\iff x\in X\mbox{ or }x\in Y.
\end{equation}
\end{definition}

Intuitively, $X\cup Y$ is the set obtained by combining the two sets $X$
and $Y$ together. We have just introduced a new term, now we need to
prove it is well-defined, i.e., we need to prove existence and
uniqueness. This is a direct consequence of a later proposition
(Proposition~\ref{prop:unions:compatible}) in this section, so we will
provisionally accept its existence and uniqueness.

\begin{theorem}[Associativity of union]
For any sets $X$, $Y$, $Z$, we have $(X\cup Y)\cup Z=X\cup(Y\cup Z)$.
\end{theorem}

This requires proving set equality, which can be done using
Proposition~\ref{prop:subsets:equiv-set-equality} to require us to prove
$(X\cup Y)\cup Z\subset X\cup(Y\cup Z)$ and $X\cup(Y\cup Z)\subset(X\cup Y)\cup Z$.
Each inclusion can be done by going back to the definitional theorem for
$\cup$.

\begin{proof}
Let $X$, $Y$, $Z$ be sets.

Claim 1: $(X\cup Y)\cup Z\subset X\cup(Y\cup Z)$. It suffices to prove,
for any object $x$, if $x\in (X\cup Y)\cup Z$, then $x\in X\cup(Y\cup Z)$.
\begin{proof}[Subproof]
Let $x$ be an object. Assume $x\in (X\cup Y)\cup Z$.
Then by Eq~\eqref{eq:def:unions:cup}, $x\in(X\cup Y)$ or $x\in Z$.
Then from $x\in X\cup Y$ and Eq~\eqref{eq:def:unions:cup}, we see $x\in X$
or $x\in Y$ or $x\in Z$.
Then applying Eq~\eqref{eq:def:unions:cup}, we see $x\in X$ or $x\in Y\cup Z$.
Hence, by applying Eq~\eqref{eq:def:unions:cup} once more, we see $x\in X\cup(Y\cup Z)$.
\end{proof}

Claim 2: $X\cup(Y\cup Z)\subset(X\cup Y)\cup Z$. Again, it suffices to
prove for any object $x$, if $x\in X\cup(Y\cup Z)$, then $x\in (X\cup Y)\cup Z$.
\begin{proof}[Subproof]
Let $x$ be an object. Assume $x\in X\cup(Y\cup Z)$.
Then by Eq~\eqref{eq:def:unions:cup}, $x\in X$ or $x\in Y\cup Z$.
Then from $x\in Y\cup Z$ and applying Eq~\eqref{eq:def:unions:cup}, we
see $x\in X$ or $x\in Y$ or $x\in Z$.
We then apply Eq~\eqref{eq:def:unions:cup} to see $x\in X\cup Y$ or
$x\in Z$.
Hence, by applying Eq~\eqref{eq:def:unions:cup} once more, we see
$x\in(X\cup Y)\cup Z$.
\end{proof}

Therefore by Proposition~\ref{prop:subsets:equiv-set-equality} and our
two claims, we conclude $(X\cup Y)\cup Z=X\cup(Y\cup Z)$.
\end{proof}

The notation for disjunction of propositions $\phi\lor\psi$ and union
of sets $A\cup B$ are similar intentionally, because if we have
$\phi[x]:=x\in A$ and $\psi[x]:=x\in B$, then $\phi[x]\lor\psi[x]$ is
logically equivalent to $x\in(A\cup B)$. In fact, the algebraic
identities of set union coincide with identities with disjunction.


\begin{xca}
Prove: for any set $X$, we have $X\cup X=X$.
\end{xca}

\begin{xca}
Prove: For any sets $X$ and $Y$, we have $X\cup Y=Y\cup X$.
\end{xca}

\begin{xca}
Prove: for any sets $X$, $Y$, $Z$, we have $(X\cup Y)\cup Z=(X\cup Z)\cup(Y\cup Z)$.
\end{xca}

\begin{xca}
Prove: For any sets $X$ and $Y$, $X\cup(X\cup Y)=X\cup Y$.
\end{xca}

\begin{xca}
Prove: for any sets $X$ and $Y$, we have $X\subset X\cup Y$.
\end{xca}

\begin{xca}
For any sets $X$, $Y$, $Z$, if $X\subset Z$ and $Y\subset Z$, then
$X\cup Y\subset Z$.
\end{xca}

\begin{xca}
For any sets $X$, $Y$, $Z$, if $X\subset Y$, then
$X\cup Z\subset Y\cup Z$.
\end{xca}

We also have a more general notion of unions, where we have a
\emph{family} (or collection) of sets $\mathcal{F}$ and we take the
union of all the members of $\mathcal{F}$.

\begin{definition}
Let $\mathcal{F}$ be a set.
We define the term $\union\mathcal{F}$ be the set satisfying
\begin{equation}\label{eq:def:unions:union-of-family}
\mbox{For all objects $x$, }x\in\union\mathcal{F}\iff\mbox{There exists a set $X$ such that }x\in X\land X\in\mathcal{F}.
\end{equation}
\end{definition}

Do you know what we need to do? First, before anything else, we should
realize we prove the existence and uniqueness of
$\union\mathcal{F}$. Before moving to that, I should bring to our
attention an axiom of ZFC set theory:

\begin{axiom}[Union]\label{axiom:existence-of-union}
For any set $\mathcal{F}$ there exists a set $A$ such that for any set
$Y$ and any object $x$, we have $x\in Y$ and $Y\in\mathcal{F}$ if and
only if $x\in A$.
\end{axiom}

\begin{theorem}[Existence]
For any set $\mathcal{F}$, there exists a set $U$ such that for all
objects $x$ we have $x\in U$ if and only if there exists a set $X$ such
that $X\in\mathcal{F}$ and $x\in X$.

In other words, for any set $\mathcal{F}$, there exists a set $U$ which
obeys the definitional theorem for $\union\mathcal{F}$.
\end{theorem}

\begin{proof}
Let $\mathcal{F}$ be a set. Thus there exists a set $U$ such that for any
set $Y$ and any object $x$, we have $x\in Y$ and $Y\in\mathcal{F}$ if
and only if $x\in U$ by Axiom~\ref{axiom:existence-of-union}.
\end{proof}

\begin{theorem}[Uniqueness]
For any set $\mathcal{F}$,
  for any sets $U_{1}$, $U_{2}$,
suppose for all objects $x_{1}$ we have $x_{1}\in U_{1}$ if and only if
there exists a set $X$ such that $X\in\mathcal{F}$ and $x_{1}\in X$;
and separately suppose for all objects $x_{2}$ we have $x_{2}\in U_{2}$
if and only if there exists a set $X$ such that $X\in\mathcal{F}$ and
$x_{2}\in X$. Then $U_{1}=U_{2}$.
\end{theorem}

In other words, all sets $U$ which satisfy the definitional theorem for
$\union\mathcal{F}$ are equal to each other. This means, there exists
exactly one set satisfying the definitional theorem for the term
$\union\mathcal{F}$. 

We should prove these two notions of unions are ``compatible''.
\begin{proposition}\label{prop:unions:compatible}
For any sets $X$ and $Y$, we have $X\cup Y=\union\{X,Y\}$.
\end{proposition}

We are proving equality of sets, we will also need to recall the
definitional-theorem for unordered pairs Eq~\eqref{enum:def:unordered-pair}.
Since we will be working with a number of definitions of the form ``for
all objects $x$, $x\in A$ iff $\mathcal{P}[x]$'', we should rely on
Eq~\eqref{eq:def:set-equality} for equality.

\begin{proof}
Let $X$ and $Y$ be sets.
For all objects $a$, $a\in X\cup Y$ if and only if $a\in\union\{X,Y\}$.
\begin{proof}[Subproof]
Let $a$ be an object.

[We will prove if $a\in X\cup Y$, then $a\in\union\{X,Y\}$.]
Assume $a\in X\cup Y$. Then $a\in X$ or $a\in Y$ by Eq~\eqref{eq:def:unions:cup}. Then there exists a
set $Z$ such that $a\in Z$ and $Z\in\{X,Y\}$ by Eq~\eqref{enum:def:unordered-pair}.
Hence $a\in\union\{X,Y\}$ by Eq~\eqref{eq:def:unions:union-of-family}.

[We will prove if $a\in\union\{X,Y\}$, then $a\in X\cup Y$.]
Assume $a\in\union\{X,Y\}$. Then there exists a set $Z$ such that
$Z\in\{X,Y\}$ and $a\in Z$ by Eq~\eqref{eq:def:unions:union-of-family}. Then, by Eq~\eqref{enum:def:unordered-pair},
either $a\in X$ or $a\in Y$. Hence $a\in X\cup Y$ by Eq~\eqref{eq:def:unions:cup}.
\end{proof}
\noindent Hence $X\cup Y=\union\{X,Y\}$ by Eq~\eqref{eq:def:set-equality}.
\end{proof}
