\section{Subsets and Set Equality}

\begin{definition}
Let $A$ and $B$ be sets. We define a new infixed predicate $A\subset B$
(read ``$A$ is a subset of $B$'') to mean:
\begin{equation}\label{eq:def:subset}
A\subset B\iff\mbox{for all objects $x$, if $x\in A$, then $x\in B$.}
\end{equation}
\end{definition}

Equation~\eqref{eq:def:subset} is a ``definitional theorem'' which states
a logical equivalence of $A\subset B$ with the formula ``for all objects
$x$, if $x\in A$, then $x\in B$''.

\begin{definition}
Let $\phi$ and $\psi$ be formulas. A \define{Logical Equivalence} of
$\psi$ with $\phi$ is a formula written $\psi\iff\phi$ (read ``$\psi$ if
and only if $\phi$''). This is an abbreviation of the formula
$(\psi\implies\phi)\land(\phi\implies\psi)$ (read ``$\psi$ implies
$\phi$, and [$\land$] $\phi$ implies $\psi$''). In practice, this means
we can replace any instance of $\psi$ by $\phi$, and vice-versa.
\end{definition}

\begin{xca}
If $\phi\iff\psi$ and $\neg\psi$ (the negation of $\psi$) are both
proven, then is $\neg\phi$ true? Write down the truth table to find out.
\end{xca}

Now, when it comes to proving things in mathematics, the basic step is
to return to the definition of the concepts involved.

\begin{definition}
Let $A$ and $B$ be sets. We define the predicate \define{Set Equality},
written $A=B$, as
\begin{equation}\label{eq:def:set-equality}
A=B\iff\mbox{for all objects $x$, we have $x\in A\iff x\in B$}.
\end{equation}
\end{definition}

\begin{remark}
This is a lie, it's not a definition, but it's an axiom. Sometimes there
is a fine line separating the two, other times they are
indistinguishable! In practice, we really don't care too much, but I
thought I should be honest about these things.
\end{remark}

OK, so we have two definitions, and we're relating them in the following
proposition:

\begin{proposition}\label{prop:subsets:equiv-set-equality}
Let $A$ and $B$ be sets.
Then $A=B$ if and only if $A\subset B$ and $B\subset A$.
\end{proposition}

What do we do? Well an ``if and only if'' statement requires proving two
claims:
\begin{enumerate}
\item If $A=B$, then $A\subset B$ and $B\subset A$
\item If $A\subset B$ and $B\subset A$, then $A=B$.
\end{enumerate}
The first case is usually indicated by writing $(\Longrightarrow)$ to
indicate which implication is being proven. We prove a formula like ``If
$\phi$, then $\psi$'' by beginning with ``Assume $\phi$ is true.'' Then
we try to infer $\psi$ somehow.

Our only two tricks in mathematics is (a) return to the definition, and
(b) assume the claim is false, then try to derive a contradiction (``If
this isn't true, nothing's true'' --- otherwise known as ``proof by
contradiction''). For us, it suffices to return to the definition, and
use the logical equivalence. I will ``think aloud'' in the square
bracketed ``asides'' and demonstrate how I think about proving things.

\begin{proof}[Proof $(\Longrightarrow)$]
Let $A$ and $B$ be a set. Assume $A=B$. [OK, great, now we return to the
definition.] Then by Eq~\eqref{eq:def:set-equality}, for all objects
$x$, we have $x\in A$ if and only if $x\in B$. [Now, we use the fact
that $\forall x,\phi\iff\psi$ is logically equivalent to
$(\forall x,\phi\implies\psi)\land(\forall x,\phi\impliedby\psi)$.]
Then every object\footnote{We contract ``For all objects $a$, if $a\in A$, then\dots''
into the more succinct statement ``Every object $a\in A$ satisfies\dots''.
This is a conventional abbreviation.}
$a\in A$ satisfies $a\in B$, and every object $b\in B$
satisfies $b\in A$. [Now we see the first part of the sentence is
  precisely saying ``For all objects $a$, if $a\in A$ then $a\in B$, 
which is the right-hand side of Eq~\eqref{eq:def:subset}; similarly, the
second part of the sentence is the right-hand side of
Eq~\eqref{eq:def:subset}. We therefore appeal to the definition of
subsets to infer:]
Then $A\subset B$ and $B\subset A$.
\end{proof}

Thus we have established, ``If $A=B$, then $A\subset B$ and $B\subset A$.''
We need to prove the opposite implication, ``If $A\subset B$ and
$B\subset A$, then $A=B$.'' This amounts to unfolding the definitions.

\begin{proof}[Proof $(\Longleftarrow)$]
Let $A$ and $B$ be sets. Assume $A\subset B$ and $B\subset A$. [Now, we
want to prove $A=B$. We begin by rewriting our assumptions using
Eq~\eqref{eq:def:subset}.] Then for all objects $a$, if $a\in A$ then
$a\in B$; and for all objects $b$, if $b\in B$ then $b\in A$. [Now we
use the logical equivalence that $(\forall a,\phi[a]\implies\psi[a])\land(\forall b,\psi[b]\implies\phi[b])\iff(\forall x,\phi[x]\iff\psi[x])$
where we write unary predicate with square brackets $\phi[x]$ and
reserve parentheses for functions.] Then for all objects $x$, if $x\in A$ then
$x\in B$, and if $x\in B$ then $x\in A$. [Now we use the definition of
$\iff$ to rewrite this formula.] Then for all objects $x$, we have $x\in A$
if and only if $x\in B$. [Now we invoke the definitional theorem Eq~\ref{eq:def:set-equality}
since our current claim is the right-hand side of the equation.]
Then $A=B$.
\end{proof}

\begin{remark}
Observe, we just produced a proof without even knowing what the
predicate $\in$ means. That's perfectly alright! Arguably, the art of
mathematics is the ability to reason about things \emph{without} know
the full details, but only working with precise specifications of
certain terms.
\end{remark}

Now, one more piece of terminology before we end this section:
\begin{definition}\label{def:proper-subset}
Let $A$ and $B$ be sets. We say $A$ is a \define{Proper Subset} of $B$
if $A\subset B$ and $A\neq B$. In this case, we write $A\propersubset B$
(in direct analogy to the notation $\leq$ and $<$).
\end{definition}

\begin{xca}\label{xca:subsets:def-thm-for-proper-subset}
What is the definitional theorem for Definition~\ref{def:proper-subset}?
\end{xca}

\begin{xca}\label{xca:proper-subsets-in-terms-of-improper-subsets}
Prove or find a counter-example: for any sets $A$ and $B$,
$A\propersubset B$ if and only if $A\subset B$ and $B\nsubseteq A$.
\end{xca}

\begin{notation}
We write ``$x\notin A$'' for ``not $x\in A$'' (or ``$\neg(x\in A)$'').
We also write ``$A\nsubset B$'' for ``not $A\subset B$'' or ``$A$ is not
a subset of $B$''. We write
``$A\neq B$'' for ``not $A=B$'' (i.e., ``$A$ is not equal to
$B$''). Similarly we write ``$A\not\propersubset B$'' for ``$A$ is not a
proper subset of $B$''.
\end{notation}

\begin{xca}\label{xca:subsets:proper-subsets-do-not-contain-some-object}
Prove or find a counter-example: for any sets $A$ and $B$, if
$A\nsubseteq B$, then there exists an object $b$ such that $b\in B$ and 
$b\notin A$. (Here we use the notation $b\notin A$ for ``not $b\in A$'')
\end{xca}

\begin{xca}\label{xca:subsets:proper-subsets-missing-something}
Prove or find a counter-example: for any sets $A$ and $B$, if
$A\propersubset B$, then there exists an object $b$ such that $b\in B$
and $b\notin A$. 
\end{xca}
