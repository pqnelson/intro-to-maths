\section{Solutions to Exercises}

\subsection{Introduction} Well, this amounts to writing down truth
tables, so you should be able to do it.

\subsection{Subsets and Set Equality} \

\begin{solution}{xca:subsets:def-thm-for-proper-subset}
  For $A\propersubset B$, its definitional theorem is technically:
\begin{equation}
\mbox{for all sets $A$ and $B$, } A\propersubset B\quad\iff\quad A\subset B\;\land\; A\neq B.
\end{equation}
We usually suppress the outermost universal quantifiers.
\end{solution}


\begin{solution}{xca:proper-subsets-in-terms-of-improper-subsets}
We will prove: For any sets $A$ and $B$, $A\propersubset B$ if and only
if $A\subset B$ and $B\nsubset A$.

\begin{proof}
Let $A$ and $B$ be sets.

Claim 1: If $A\propersubset B$, then $A\subset B$ and $B\nsubset A$.
\begin{proof}[Subproof of Claim 1]
  Assume $A\propersubset B$. Then
  \begin{subequations}
    \begin{equation}\label{soln:bool:proper-subsets-in-terms-of-improper-subsets:step1}
A\subset B
    \end{equation}
    and
    \begin{equation}
A\neq B
    \end{equation}
by Definition~\ref{def:proper-subset}.
  \end{subequations}
But $A\neq B$ is logically equivalent to not ($A\subset B$ and $B\subset A$)
by Proposition~\ref{prop:subsets:equiv-set-equality}.
Then either $A\nsubset B$ or $B\nsubset A$ by de Morgan's law.
But we have $A\subset B$ by Eq~\eqref{soln:bool:proper-subsets-in-terms-of-improper-subsets:step1}, therefore we are forced to conclude $B\nsubset A$.
Hence we have proved $A\subset B$ and $B\nsubset A$.
\end{proof}
Claim 2: If $A\subset B$ and $B\nsubset A$, then $A\propersubset B$.
\begin{proof}[Subproof of Claim 2]
Assume $A\subset B$ and $B\nsubset A$. We want to prove $A\subset B$ and
$A\neq B$ by Definition~\ref{def:proper-subset}. But we assumed
$A\subset B$, so we really just need to prove $A\neq B$.
We can use Proposition~\ref{prop:subsets:equiv-set-equality} to use the
logical equivalence $A\neq B$ if and only if either $A\nsubset B$ or
$B\nsubset A$. We assumed $B\nsubset A$. Therefore we can infer $A\neq B$.
Hence we find $A\propersubset B$.
\end{proof}
Thus we have proved $A\propersubset B$ if and only if $A\subset B$ and
$B\nsubset A$ by Claims 1 and 2.
\end{proof}
\end{solution}


\begin{solution}{xca:subsets:proper-subsets-do-not-contain-some-object}
\textit{Prove or find a counter-example: for any sets $A$ and $B$, if
$A\nsubseteq B$, then there exists an object $b$ such that $b\in B$ and 
$b\notin A$. (Here we use the notation $b\notin A$ for ``not $b\in A$'')}

Let $A$ be any nonempty set. There exists a proper subset
$B\propersubset A$ ($B$ could be the empty set).
Consider $B\propersubset A$.
Then $A\nsubseteq B$ and there is no $b\in B$
such that $b\notin A$.

So we found a counter-example to the claim.
\end{solution}


\begin{solution}{xca:subsets:proper-subsets-missing-something}
\textit{Prove or find a counter-example: for any sets $A$ and $B$, if
$A\propersubset B$, then there exists an object $b$ such that $b\in B$
and $b\notin A$.}

Let us try to prove it directly.
Let $A$ and $B$ be sets. Assume $A\propersubset B$. Then by previous exercises, $A\subset B$ and $B\nsubseteq A$.

Then $B\nsubseteq A$.

Then not $B\subset A$.

Then not (for any objects $x$, if $x\in B$, then $x\in A$) by Definition~\eqref{eq:def:subset}.

Then there exists an object $x$ such that not (if $x\in B$ then $x\in A$)
by de Morgan's law.

Then there exists an object $x$ such that not (either not $x\in B$, or
$x\in A$) by ($\phi\implies\psi$ is logically equivalent to $\neg\phi\lor\psi$).

Then there exists an object $x$ such that both $x\in B$ and $x\notin A$
by de Morgan's other law, and double negation.

Hence there exists an object $x$ such that $x\in B$ and $x\notin A$.
\end{solution}


\subsection{Enumerated Sets} \

\begin{solution}{xca:enumerated-sets:singleton-subsets-of-singleton}
For all objects $x$ and $y$, if $\{x\}\subset\{y\}$, then $x=y$.

\begin{proof}
Let $x$ and $y$ be objects. Assume $\{x\}\subset\{y\}$. [Now we unfold
the definition of $\subset$.]
Then by Eq~\eqref{eq:def:subset}, for all objects $z$, if $z\in\{x\}$,
then $z\in\{y\}$. [Now, we unfold the definition of singleton set for the premise.]
Then by Eq~\eqref{enum:def:singleton}, for all objects $z$, if $z=x$,
then $z\in\{y\}$.
Then we have $x\in\{y\}$.
Then by Eq~\eqref{enum:def:singleton}, $x=y$.
Hence we have obtained $x=y$.
\end{proof}
\end{solution}

\begin{solution}{xca:enum:singleton-subset-unordered-pair}
For all objects $x$ and $y$, we have $\{x\}\subset\{x,y\}$.
\begin{proof}
Let $x$, $y$ be objects.
For any object $z$, if $z\in\{x\}$, then $z\in\{x,y\}$.
\begin{proof}[Subproof]
Let $z$ be an object. Assume $z\in\{x\}$. Then $z=x$ by Eq~\eqref{enum:def:singleton}.
Then $z=x$ or $z=y$.
Hence $z\in\{x,y\}$ by Eq~\eqref{enum:def:unordered-pair}.
\end{proof}
\noindent Hence $\{x\}\subset\{x,y\}$ by Eq~\eqref{eq:def:subset}.
\end{proof}
\end{solution}

\subsection{Boolean Properties of Sets } \

\begin{solution}{xca:bool:empty-is-adjective}
We see that ``empty'' is an adjective of sets. In general, whenever a
definition takes the form ``Let $t$ be a [type] $T$. We say $t$ is
$\alpha$ to mean\dots'', it's almost certainly defining an adjective.

Another heuristic is to see if it makes sense to say aloud ``non-$\alpha$''.
If so, then it's probably an adjective.
\end{solution}

\begin{solution}{xca:bool:subsets-of-empty-set-are-empty}
For any set $X$, if $X\subset\emptyset$, then $X=\emptyset$.

\begin{proof}
Let $X$ be a set. Assume $X\subset\emptyset$. [We will use
Proposition~\ref{prop:subsets:equiv-set-equality} to change the proof
goal from $X=\emptyset$ to $X\subset\emptyset$ and $\emptyset\subset X$.
Since we have already assumed $X\subset\emptyset$, we see our proof goal
requires us to prove $\emptyset\subset X$.]
We have $\emptyset\subset X$ by Lemma~\ref{lemma:bool:empty-set-is-subset-of-any-set}.
Then we have $X\subset\emptyset$ and $\emptyset\subset X$.
Hence we conclude $X=\emptyset$ by Proposition~\ref{prop:subsets:equiv-set-equality}.
\end{proof}
\end{solution}


\begin{solution}{xca:bool:modus-barbara}
We want to prove ``For any sets $X$, $Y$, and
$Z$, if $X\subset Y$ and $Y\subset Z$, then $X\subset Z$.''

Let $X$, $Y$, $Z$ be sets. Assume $X\subset Y$ and $Y\subset Z$.
We want to prove $X\subset Z$. [This requires returning to the
Definition of subset, specifically the definitional theorem Eq~\eqref{eq:def:subset}.
We will want to prove ``For all objects $x$, if $x\in X$, then $x\in Z$''.]
So for all objects $x$, if $x\in X$ then $x\in Y$; and for all objects
$y$, if $y\in Y$, then $y\in Z$.

We claim for all objects $x$, if $x\in X$, then $x\in Z$.
\begin{proof}
Let $x$ be an
arbitrary object. Assume $x\in X$. Then by assumption $X\subset Y$, we
see $x\in Y$ using Eq~\eqref{eq:def:subset}. Then using assumption
$Y\subset Z$, we see $x\in Z$ using Eq~\eqref{eq:def:subset}. Hence $x\in Z$
as desired. This proves the claim.
\end{proof}
Then we use Eq~\eqref{eq:def:subset} to conclude $X\subset Z$.
Hence we prove the result.
\end{solution}

%% \begin{solution}{xca:bool:modus-barbara}
%% For any sets $X$, $Y$, $Z$, if $X\subset Y$ and $Y\subset Z$, then
%% $X\subset Z$.

%% \begin{proof}
%%   Let $X$, $Y$, $Z$ be sets. Assume
%%   \begin{subequations}
%%     \begin{equation}
%% X\subset Y
%%     \end{equation}
%% and
%% \begin{equation}
%% Y\subset Z.
%% \end{equation}
%%   \end{subequations}
%%   For any object $x$, if $x\in X$, then $x\in Z$.
%%   \begin{proof}[Subproof]
%% Let $x$ be an object. Assume $x\in X$. Since $X\subset Y$, then by Eq~\eqref{eq:def:subset}
%% $x\in Y$. Then, since $Y\subset Z$, we find $x\in Z$ by Eq~\eqref{eq:def:subset}.
%% Hence we conclude $x\in Z$.
%%   \end{proof}
%% \noindent Hence we conclude $X\subset Z$ by Eq~\eqref{eq:def:subset}.
%% \end{proof}
%% \end{solution}
