\section{Intersections of Sets}

Dual to unions ``combining'' sets together, we can consider the
intersection or ``largest common subset'' between two sets. Before
jumping to the fun parts, we need to introduce some preliminary notions.

We begin by introducing a new variant of a theorem called a
\define{scheme}. It's like a theorem, but we allow arbitrary predicates
or terms which we indicate using caligraphic font (e.g., $\mathcal{P}$,
$\mathcal{F}$, $\mathcal{X}$, etc.). This is an axiom scheme for set
theory. 

\begin{axiom-scheme}[Replacement]\label{sch:bool:replacement}
Let $\mathcal{P}[-]$ be a unary predicate of sets, let $\mathcal{A}$ be
a set.
There exists a set $X$ such that for all objects $x$ we have $x\in X$ if
and only if $x\in\mathcal{A}$ and $\mathcal{P}[x]$.
\end{axiom-scheme}

This is usally taken as an axiom in ZFC set theory. In
Tarski--Grothendieck set theory, it can be derived as a theorem. We will
accept it as an axiom: for any set $\mathcal{A}$ and predicate
$\mathcal{P}[-]$, we can form a subset of $\mathcal{A}$ consisting of
its elements which satisfy $\mathcal{P}[-]$. This makes sense.

\begin{definition}
Let $A$ and $B$ be sets. We define the term $A\cap B$ to be the set
satisfying
\begin{equation}
\mbox{For all objects $x$, }x\in A\cap B\iff x\in A\mbox{ and }x\in B.
\end{equation}
\end{definition}

Having just defined a new term, we now must prove existence and
uniqueness.

\begin{theorem}[Existence of $\cap$]
Let $A$ and $B$ be sets. There exists a set $X$ such that for all
objects $x$ we have $x\in X$ if and only if $x\in A$ and $x\in B$.
\end{theorem}

We can use Axiom Scheme~\ref{sch:bool:replacement} to consider the
subset of $A$ consisting of elements of $B$.

\begin{proof}
Let $A$ and $B$ be sets. We define the predicate $\mathcal{P}[x]:=x\in B$.
Then there exists a set $X$ such that for all objects $x$ we have $x\in X$
if and only if $x\in A$ and $\mathcal{P}[x]$ (i.e., $x\in A$ and $x\in B$).
Hence there exists a set satisfying the definitional theorem for set
intersections. 
\end{proof}

\begin{theorem}[Uniqueness of $\cap$]
Let $A$ and $B$ be sets.
For any sets $X$, $Y$ such that (i) for all objects $x$ we have $x\in X$
if and only if $x\in A$ and $x\in B$, and (ii) for all objects $y$ we
have $y\in Y$ if and only if $y\in A$ and $y\in B$;
then we have $X=Y$.
\end{theorem}

\begin{proof}
Let $A$ and $B$ be sets. Let $X$ and $Y$ be sets. Assume
\begin{subequations}
  \begin{equation}
\mbox{for all objects $x$ we have $x\in X$ if and only if $x\in A$ and $x\in B$.}
  \end{equation}
  Assume
  \begin{equation}
\mbox{for all objects $y$ we have $y\in Y$ if and only if $y\in A$ and $y\in B$}.
  \end{equation}
\end{subequations}
[We will use the definitional-theorem Eq~\eqref{eq:def:set-equality} for
set equality.]
Now let $x$ be an arbitrary object. We have $x\in X$ iff $x\in A$ and
$x\in B$ by assumption. Then we have $x\in X$ iff $x\in Y$.
Since this is true for arbitrary $x$, we have just established ``For any
object $x$, we have $x\in X$ iff $x\in Y$''. Therefore, by Eq~\eqref{eq:def:set-equality},
we have $X=Y$.
\end{proof}

\begin{proposition}
For any sets $X$, $Y$, $Z$ we have $X\cap(Y\cap Z)=(X\cap Y)\cap Z$.
\end{proposition}

\begin{xca}
Prove: for any set $X$, we have $X\cap X=X$.
\end{xca}

\begin{xca}
Prove: For any sets $X$ and $Y$, we have $X\cap Y=Y\cap X$.
\end{xca}

\begin{xca}
Prove or find a counter-example: for any sets $X$, $Y$, $Z$, if
$X\subset Y\cap Z$, then $X\subset Y$.
\end{xca}

\begin{xca}
Prove or find a counter-example: for any sets $X$, $Y$, $Z$, if
$Z\subset X$ and $Z\subset Y$, then $Z\subset X\cap Y$.
\end{xca}

\begin{xca}
Prove or find a counter-example: for any sets $X$, $Y$, $Z$,
we have $X\cap(Y\cup Z)=(X\cap Y)\cup(X\cap Z)$.
\end{xca}

\begin{xca}
Prove or find a counter-example: for any sets $X$, $Y$, $Z$,
we have $X\cup(Y\cap Z)=(X\cup Y)\cap(X\cup Z)$.
\end{xca}

\begin{xca}
Prove or find a counter-example: for any sets $X$ and $Y$, if $X\subset Y$,
then $X\cap Y=X$.
\end{xca}
