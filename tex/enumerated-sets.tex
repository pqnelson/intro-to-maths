\section{Enumerated Sets}

We begin by noting there exists a pair. This is taken as an axiom of ZFC
set theory (and all set theories mentioned in the introduction).

\begin{axiom}[Pair]\label{axiom:pair}
For any sets $X$ and $Y$, there exists a set $Z$ such that for all
objects $x$ we have $x\in Z$ if and only if $x=X$ or $x=Y$.
\end{axiom}

\begin{definition}
Let $y$ be any object. We define the term $\{y\}$ to be the set
satisfying
\begin{equation}\label{enum:def:singleton}
\mbox{for all objects $x$, we have }x\in\{y\}\iff x=y.
\end{equation}
\end{definition}

We need to prove the existence and uniqueness of $\{y\}$.

\begin{theorem}[Existence]
For any object $x$, there exists a set $A$ such that for all objects $y$
we have $y\in B$ if and only if $y=x$.
\end{theorem}

In other words, there exists at least one set $B$ which behaves like $\{x\}$.
This is actually proven immediately by Axiom~\ref{axiom:pair}.

\begin{proof}
Let $x$ be an object. Consider $A$ being a set such that
\begin{equation}
\mbox{For all objects $y$, }y\in A\iff y=x\mbox{ or }y=x
\end{equation}
by Axiom~\ref{axiom:pair}. Take $A$.
Thus for all objects $y$ we have $y\in A$ if and only if $y=x$.
\end{proof}

\begin{theorem}[Uniqueness]
Let $x$ be any object. For all sets $A$ and $B$ such that for all
objects $y_{1}$ we have $y_{1}\in A$ if and only if $y_{1}=x$, and
separately for all objects $y_{2}$ we have $y_{2}\in B$ if and only if $y_{2}=x$.
Then $A=B$.
\end{theorem}

This establishes all sets behaving like $\{x\}$ are equal to each other. 

\begin{proof}
Let $x$ be an object. Let $A$ and $B$ be sets. Assume
\begin{subequations}\label{eq:uniqueness-of-singleton}
\begin{equation}
\mbox{For all objects $y_{1}$, }y_{1}\in A\iff y_{1}=x.
\end{equation}
Assume
\begin{equation}
\mbox{For all objects $y_{2}$, }y_{2}\in B\iff y_{2}=x.
\end{equation}
\end{subequations}
[Now to prove $A=B$, we just use the definitional theorem Eq~\eqref{eq:def:set-equality}.]
For all objects $y$, we have $y\in A$ iff $y\in B$ by Eq~\eqref{eq:uniqueness-of-singleton}.
Hence $A=B$ by Eq~\eqref{eq:def:set-equality}.
\end{proof}

\begin{definition}
Let $y$ and $z$ be objects. We define the term $\{y,z\}$ to be the set
satisfying
\begin{equation}\label{enum:def:unordered-pair}
\mbox{for all objects $x$, we have }x\in\{y,z\}\iff x=y\mbox{ or }x=z.
\end{equation}
\end{definition}

We need to prove existence and uniqueness of $\{x,y\}$ for any objects
$x$ and $y$. Existence follows immediately from Axiom~\ref{axiom:pair},
quite literally they are identical statements. Therefore we only need to
prove uniqueness.

\begin{theorem}[Uniqueness]
Let $x$, $y$ be objects. If for all sets $A$ and $B$ such that
(i) for all objects $a$ we have $a\in A$ iff $a=x$ or $a=y$, and
separately (ii) for all objects $b$ we have $b\in B$ iff $b=x$ or $b=y$;
then $A=B$.
\end{theorem}

We are proving two sets are equal. Since the sets we want to prove are
equal to each other are defined using the templace ``For all objects $w$,
we have $w\in X$ iff $\mathcal{P}[w]$'' for some predicate $\mathcal{P}[-]$,
it will be advantageous to use the definitional-theorem Eq~\eqref{eq:def:set-equality}.

\begin{proof}
Let $x$, $y$ be objects. Let $A$ and $B$ be sets. Assume
\begin{subequations}\label{enumerated-set:pf:uniqueness-pair}
\begin{equation}
\mbox{for all objects $a$ we have $a\in A$ iff $a=x$ or $a=y$}.
\end{equation}
Assume
\begin{equation}
\mbox{for all objects $b$ we have $b\in B$ iff $b=x$ or $b=y$}.
\end{equation}
\end{subequations}
[We want to prove $A=B$.]
Now let $z$ be an arbitrary object. We have $z\in A$ iff $z=x$ or $z=y$
by assumption.
Then we have $z\in B$ iff $z\in A$ by assumptions~\eqref{enumerated-set:pf:uniqueness-pair}.
Hence $A=B$ by definitional-theorem Eq~\eqref{eq:def:set-equality}.
\end{proof}

\begin{proposition}
For any object $x$, we have $\{x\}=\{x,x\}$.
\end{proposition}

OK, what are we trying to prove? Ultimately, we are trying to prove two
sets are equal to each other. This requires invoking the definition of
set equality as stated in Eq~\eqref{eq:def:set-equality} or the
equivalent criterion found in
Proposition~\ref{prop:subsets:equiv-set-equality} (but this will, in
turn, require invoking Eq~\ref{eq:def:subset} for the definition of subsets).

\begin{proof}
Let $x$ be any object. [Now we want to prove $\{x\}=\{x,x\}$, which if
we used Eq~\eqref{eq:def:set-equality} transforms our obligation into
the claim ``For all objects $y$, we have $y\in\{x\}$ if and only if
$y\in\{x,x\}$''.]
Let $y$ be an arbitrary object. We will prove $y\in\{x\}$ if and only if
$y\in\{x,x\}$.

Assume $y\in\{x\}$. Then by Eq~\eqref{enum:def:singleton}, $y=x$. [Then
$y=x$ or $y=x$.] Hence by Eq~\eqref{enum:def:unordered-pair}, $y\in\{x,x\}$
[This proves the claim, ``If $y\in\{x\}$, then $y\in\{x,x\}$.'']

Now assume $y\in\{x,x\}$. [We will prove $y\in\{x\}$.]
Then by Eq~\eqref{enum:def:unordered-pair}, $y=x$ or $y=x$. [But since
these are identical conditions, we can use the logical equivalence
$(A\lor A)\iff A$ and obtain:] Then $y=x$. Hence by Eq~\eqref{enum:def:singleton}
$y\in\{x\}$ as desired. [This proves the claim, ``If $y\in\{x,x\}$,
then $y\in\{x\}$''.]

Since $y$ has been left arbitrary, these two paragraphs establish the
claim ``For any object $y$, we have $y\in\{x\}$ if and only if $y\in\{x,x\}$.''
Therefore, by Eq~\eqref{eq:def:set-equality}, we conclude $\{x\}=\{x,x\}$
as desired.
\end{proof}

\begin{xca}
Can you offer a definition for a set $\{x,y,z\}$ for any objects $x$,
$y$, and $z$?
\end{xca}

\begin{xca}\label{xca:enumerated-sets:singleton-subsets-of-singleton}
Prove or find a counter-example: For all objects $x$ and $y$, if
$\{x\}\subset\{y\}$, then $x=y$.
\end{xca}

\begin{xca}\label{xca:enum:singleton-subset-unordered-pair}
Prove or find a counter-example: For all objects $x$ and $y$, we have $\{x\}\subset\{x,y\}$.
\end{xca}
